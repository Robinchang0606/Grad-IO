\documentclass[10pt,aspectratio=169]{beamer}

% silence some Metropolis warnings
\usepackage{silence}
\WarningFilter{beamerthememetropolis}{You need to compile with XeLaTeX or LuaLaTeX}
\WarningFilter{latexfont}{Font shape}
\WarningFilter{latexfont}{Some font}

% define custom colors
\usepackage{xcolor}
\definecolor{dark gray}{HTML}{444444}
\definecolor{light gray}{HTML}{777777}
\definecolor{dark red}{HTML}{BB0000}
\definecolor{dark green}{HTML}{00BB00}
\definecolor{RoyalBlue}{cmyk}{1, 0.50, 0, 0}

% configure metropolis
\usetheme[numbering=fraction]{metropolis}
\setbeamercolor{background canvas}{bg=white}
\setbeamercolor{frametitle}{bg=dark gray}
\setbeamercolor{alerted text}{fg=dark red}
\setbeamercolor{item projected}{bg=dark red}
\setbeamercolor{local structure}{fg=dark red}
\setbeamersize{text margin left=0.5cm,text margin right=0.5cm}
\setbeamercovered{transparent=10}

% use thicker lines
\makeatletter
\setlength{\metropolis@titleseparator@linewidth}{1pt}
\setlength{\metropolis@progressonsectionpage@linewidth}{1pt}
\makeatother

% custom bullet points
\setbeamertemplate{itemize item}{\color{dark red}$\blacktriangleright$}
\setbeamertemplate{itemize subitem}{\color{dark red}$\blacktriangleright$}
\setbeamertemplate{itemize subsubitem}{\color{dark red}$\blacktriangleright$}
\newcommand{\custombullet}{{\color{dark red}$\blacktriangleright$}\hspace{0.5em}}


% imports
\usepackage[english]{babel}
\usepackage[utf8]{inputenc}
\usepackage{amsthm}
\usepackage{amssymb}
\usepackage{amsmath}
\usepackage{amsfonts}
\usepackage{mathtools}
\usepackage{mathabx}
\usepackage{stmaryrd}
\usepackage{graphicx}
\usepackage{hyperref}
\usepackage{xfrac}
\usepackage{appendixnumberbeamer}
\usepackage{tabularx}

% check and x marks
\usepackage{pifont}
\newcommand{\cmark}{{\color{dark green}\ding{51}}\hspace{0.3em}}
\newcommand{\xmark}{{\color{dark red}\ding{55}}\hspace{0.5em}}


% use classic font for math
\usepackage[T1]{fontenc} % Needed for Type1 Concrete \usepackage{concmath}
\usefonttheme{serif}
\usefonttheme{professionalfonts}
\usepackage{concmath}
\setbeamerfont{equation}{size=\tiny}



% diagrams
\usepackage{tikz}
\usetikzlibrary{decorations.pathreplacing}

% references
\usepackage[natbibapa]{apacite}
\bibliographystyle{apacite}
\renewcommand{\bibsection}{}

% use ampersands instead of "and" for text citations
\AtBeginDocument{\renewcommand{\BBAB}{\&}}

% possessive cites
\makeatletter
\patchcmd{\NAT@test}{\else \NAT@nm}{\else \NAT@nmfmt{\NAT@nm}}{}{}
\DeclareRobustCommand\citepos
  {\begingroup
   \let\NAT@nmfmt\NAT@posfmt
   \NAT@swafalse\let\NAT@ctype\z@\NAT@partrue
   \@ifstar{\NAT@fulltrue\NAT@citetp}{\NAT@fullfalse\NAT@citetp}}
\let\NAT@orig@nmfmt\NAT@nmfmt
\def\NAT@posfmt#1{\NAT@orig@nmfmt{#1's}}
\makeatother

% spaced-out lists
\newenvironment{wideitemize}{\itemize\addtolength{\itemsep}{10pt}}{\enditemize}
\newenvironment{wideenumerate}{\enumerate\addtolength{\itemsep}{10pt}}{\endenumerate}

% replace footnotes with buttons
\usepackage[absolute,overlay]{textpos}
\newcounter{beamerpausessave}
\newcommand{\always}[1]{
    \setcounter{beamerpausessave}{\value{beamerpauses}}
    \setcounter{beamerpauses}{0}
    \pause
    #1 
    \setcounter{beamerpauses}{\value{beamerpausessave}}
    \addtocounter{beamerpauses}{-1}
    \pause
}
\newcommand{\buttons}[1]{\always{
    \begin{textblock*}{\paperwidth}(0.015\textwidth, 1.022\textheight)
        \scriptsize
        #1
    \end{textblock*}
}}
\newcommand{\appendixbuttons}[1]{\always{
    \begin{textblock*}{\paperwidth}(0.015\textwidth, 1.043\textheight)
        \scriptsize
        #1
    \end{textblock*}
}}
\newcommand{\goto}[2]{\hyperlink{#1}{{\color{dark red}$\smalltriangleright$} #2}\hspace{0.5em}}
\newcommand{\goback}[2]{\hyperlink{#1}{{\color{dark red}$\smalltriangleleft$} #2}\hspace{0.5em}}

% custom appendix
\renewcommand{\appendixname}{\texorpdfstring{\translate{Appendix}}{Appendix}}

% change color of cites and URLs
\let\oldcite\cite
\let\oldcitet\citet
\let\oldcitep\citep
\let\oldcitepos\citepos
\let\oldcitetalias\citetalias
\let\oldcitepalias\citepalias
\let\oldurl\url
\def\cite#1#{\citeaux{#1}}
\def\citet#1#{\citetaux{#1}}
\def\citep#1#{\citepaux{#1}}
\def\citepos#1#{\citeposaux{#1}}
\def\citetalias#1#{\citetaliasaux{#1}}
\def\citepalias#1#{\citepaliasaux{#1}}
\def\url#1#{\urlaux{#1}}
\newcommand*\citeaux[2]{{\color{light gray}\oldcite#1{#2}}}
\newcommand*\citetaux[2]{{\color{light gray}\oldcitet#1{#2}}}
\newcommand*\citepaux[2]{{\color{light gray}\oldcitep#1{#2}}}
\newcommand*\urlaux[2]{{\color{light gray}\oldurl#1{#2}}}
\newcommand*\citeposaux[2]{{\color{light gray}\oldcitepos#1{#2}}}
\newcommand*\citetaliasaux[2]{{\color{light gray}\oldcitetalias#1{#2}}}
\newcommand*\citepaliasaux[2]{{\color{light gray}\oldcitepalias#1{#2}}}

% custom math commands
\DeclareMathOperator*{\argmax}{argmax}
\DeclareMathOperator*{\argmin}{argmin}
\renewcommand{\Pr}{\mathbb{P}}
\newcommand{\E}{\mathbb{E}}
\newcommand{\Var}{\mathbb{V}}
\newcommand{\Cov}{\mathbb{C}}
\newcommand{\overbar}[1]{\mkern 1.5mu\overline{\mkern-1.5mu#1\mkern-1.5mu}\mkern 1.5mu}

% tables
\usepackage{booktabs}
\usepackage{colortbl}
\usepackage{multirow}
\usepackage{makecell}
\arrayrulecolor{dark red}

% custom date
\usepackage{datetime}
\newdateformat{monthyeardate}{\monthname[\THEMONTH] \THEYEAR}

% fix pauses with graphics
\usepackage{../resources/fixpauseincludegraphics}



\title [Course Intro]{Course Introduction: Empirical IO}
\author{C.Conlon}
\institute{Grad IO }
\date{Fall 2024}
\begin{document}

\begin{frame}
\titlepage
\end{frame}

\begin{frame}{Course Requirements}
This course will consist of:
\begin{itemize}
    \item Weekly readings/Participation
    \item 4-5 Problem Sets (15\% each)
    \item A research proposal (20\%).
\end{itemize}
\end{frame}


\begin{frame}{Preparation}
Most students are in the 2nd year of their PhD. For the course to make sense I assume you are familiar with:
\begin{itemize}
    \item Maximum Likelihood and GMM estimation
    \item First-year micro consumer theory (taking first-order conditions, deriving comparative statics)
    \item Some basic game theory (what is Nash Eq? What is subgame perfection?)
    \item Undergraduate IO at the level of \href{https://www.amazon.com/Introduction-Industrial-Organization-MIT-Press/dp/0262035944}{[Cabral's book]}
    \item Some familiarity with Tirole-style IO modeling \href{https://www.amazon.com/Theory-Industrial-Organization-MIT-Press/dp/0262200716/}{[Tirole's book]}
\end{itemize}
\end{frame}



\begin{frame}{Books}
You are not required to go out and buy textbooks (We don't specifically follow any one). But here are some books most IO economists own and find useful:
\begin{itemize}
    \item Train (2009): Discrete Choice Methods with Simulation \url{https://eml.berkeley.edu/books/choice2.html}
    \item ``Discrete Choice Theory of Product Differentiation'' by Anderson, De Palma, Thisse (now out of print)
    \item Vives (2001): Oligpoly Pricing: Old Ideas and New Tools \href{https://www.amazon.com/Oligopoly-Pricing-Old-Ideas-Tools/dp/026272040X}{[Vives]}
    \item Deaton and Muellbauer (1980): Economics and Consumer Behavior \href{https://www.amazon.com/Economics-Consumer-Behavior-Angus-Deaton/dp/0521296765}{[Deaton Muellbauer]}
\end{itemize}
\end{frame}

\begin{frame}{Questions}
\begin{itemize}
    \item Can you audit the course? Yes, but I would prefer you take the course for credit.
    \item Should I take this course in my first year? No, I teach this course every year.
    \item If I want to do IO as my primary field, what else do I need to do?
    \begin{itemize}
        \item Take Prof Jovanovic's IO Theory course
        \item Take Prof Waldinger's Empirical IO Course in the Spring
        \item Attend the Friday IO Seminar at 11am
        \item Attend the Friday Stern Workshop at 4pm
        \item Eventually find three people to serve on your committee
    \end{itemize}
\end{itemize}
\end{frame}



\begin{frame}{IO at NYU}
This is is now one of the largest IO groups in the world
\begin{itemize}
    \item Luis Cabral (Stern): Applied IO Theory, Innovation, Platforms, Signalling, Reputation, Creative Industries
    \item Michael Dickstein (Stern): IO of Healthcare Markets (Hospitals, Physicians, Insurers, Pharma) and IO/International Trade
    \item Paul Scott (Stern): IO of Agriculture and Land Use, IO and Environmental Economics, Telecom, Production and Markups
    \item Giulia Brancaccio (Stern): IO of Transportation Markets, IO and International Trade, Markets with Search and Matching Frictions, IO/Finance
    \item Daniel Waldinger (FAS): Empirical Market Design, IO and Urban Economics (Rent Control, Public Housing, etc.)
    \item Audrey Tiew (FAS): Antitrust, Dynamic Games, Semiconductors, Environmental IO
    \item Thi Mai Anh Nguyen (FAS): Search and Matching (Trucking), IO and Transportation
    \item Anna Vitali (FAS): Development/IO (Entry, Firm Location, Consumer Search)
\end{itemize}
\end{frame}

\begin{frame}{IO as a set of Tools}
\begin{itemize}
    \item Tools for estimating demand in multi-product settings
    \item Tools for analyzing strategic interactions (prices, quantities, investment, quality) among firms
    \item How do changes (subsidies, taxes, tariffs, entry barriers) change market outcomes
    \item Modeling search and matching frictions
    \item Modeling contractual incentives (moral hazard, adverse selection, vertical contracts)
\end{itemize}
\end{frame}



\begin{frame}{IO as a set of Tools}
Lots of recent students have done ``IO-Plus''
\begin{itemize}
    \item Angela Crema (Rochester): Race-preferences and School Quality after School Choice
    \item Pierre Bodere (Yale SOM): Entry and Quality Competition in Pre-Schools
    \item Jonathan Elliott (Johns Hopkins): Electricity Generation and Green Subsidy Design in Western Australia
    \item German Guittierez (Washington Foster): Platform Design and Competition on Amazon
    \item Yinan Wang (Amazon): Algorithmic Pricing on Airbnb
    \item Chitra Marti (Cornerstone): Economics of Cybersecurity
    \item Chiara Gardenghi: Bundled Discounts in Physician Administered Drugs
    \item Nano Ochoa: Housing Subsidy Design in Chile
    \item Helena Pedrotti: Social Housing Construction in France
\end{itemize}
\end{frame}



\begin{frame}{IO as a set of questions}
\begin{itemize}
    \item What is market power and where does it come from?
    \item Who competes with whom?
    \item What features of markets tell us about competition, innovation, product quality, etc?
    \item How can we tell if firms are competing (or colluding) from data?
    \item How can we design mechanisms using data (auctions, kidneys, etc.)
    \item 
\end{itemize}
\end{frame}







\end{document}













































